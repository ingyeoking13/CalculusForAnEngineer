\documentclass{article}
\usepackage{kotex}
\usepackage{amsmath, amssymb}
\begin{document}
\setcounter{section}{2}
\section{도함수의 연산}

\subsection{다항식의 도함수}

함수의 도함수는 정의는 다음과 같다.
\[
    f'(x) = \lim_{\Delta x \to 0}\frac{f(x+\Delta x)-f(x)}{\Delta x},
\]
또는
\[
    \frac{dy}{dx}=\frac{d}{dx}f(x) = \lim_{\Delta x \to 0}\frac{\Delta y}{\Delta x}
\]
이번 섹션에서는 도함수 연산을 연습한다. 
다항식에 대한 미분의 간략한 규칙은 다음과 같다.\\\\
\textit{1. 상수의 도함수는 0이다.}
\[
    \frac{d}{dx}c = 0.
\]
기하학적 의미로는 x축에 대한 수평선 $y=f(x)=c$는 기울기 $m=0$, $\forall x$이라는 뜻이다.
\\\\
\textit{2. $n$이 양의 정수이면, 다음을 만족한다.}  
\[
    \frac{d}{dx}x^{n}=nx^{n-1}.
\]
$x^n$의 도함수는 지수 $n$을 내려오게하여 상수 계수로 만들고, 기존 지수에서 $1$을 차감하여 새 지수로 만든다.
증명은 다음과 같다.
\begin{align*}
    \text{Let } f(x)&=x^n. \\
    \Delta y&= f(x+\Delta x) - f(x) = (x+\Delta x)^n - \Delta x^n \\
            &= \Big{[}x^n+nx^{n-1}+\frac{n(n-1)}{2}x^{n-2}(\Delta x)^2 + ... + (\Delta x)^n \Big{]} - x^n \\
            &= nx^{n-1}\Delta x + \frac{n(n-1)}{2}x^{n-2}(\Delta x)^2 + ... + (\Delta x)^n.\\
    \text{and } \\
    \frac{dy}{dx} &= \lim_{\Delta x \to 0}\frac{\Delta y}{\Delta x}\\
    &= \lim_{\Delta x \to 0}\Big{[}nx^{n-1} + \frac{n(n-1)}{2}x^{n-2}\Delta x + ... + (\Delta x)^{n-1}\Big{]}\\
    &= nx^{n-1}
\end{align*}
\\\\
\textit{3. $c$가 상수이고 $u=f(x)$가 x에 대해 미분가능할 때, 다음을 만족한다.}\\  
\[
    \frac{d}{dx}(cu)=c\frac{du}{dx}
\]
즉, 임의의 함수의 상수배인 함수는 도함수도 상수배이다. 이를 증명하기 위해 $y=cu=cf(x)$라 하자. 
그렇다면 $\Delta y = cf(x+\Delta x)-cf(x)=c[f(x+\Delta x)-f(x)]=c\Delta u$ 이다. 이 때 다음을 만족한다.  
\[
    \frac{dy}{dx}=\lim_{\Delta x \to 0}\frac{\Delta y}{\Delta x} = \lim_{\Delta x \to 0}\frac{c\Delta u}{\Delta x}=c\lim_{\Delta x \to 0}\frac{\Delta u}{\Delta x}=c\frac{\Delta u}{\Delta x} 
\]
규칙 2와 3을 통해 다음을 확인할 수 있다.  
\[
    \frac{d}{dx}cx^n=cx^{n-1}
\]
\\\\
\textit{4. $u=f(x)$와 $v=g(x)$가 $x$의 함수일 때, 다음을 만족한다.}
\[
    \frac{d}{dx}(u+v)=\frac{du}{dx}+\frac{dv}{dx}.
\]
증명은 다음을 따른다. $y= u+v = f(x)+g(x)$라 하자. 그렇다면 $\Delta y=[f(x+\Delta x)+g(x+\Delta x)]-[f(x)+g(x)]= [f(x+\Delta x)-f(x)] + [g(x + \Delta x)-g(x)] = \Delta u + \Delta v$
\begin{align*}
    \frac{dy}{dx}&=\lim_{\Delta x \to 0}\frac{\Delta y}{\Delta x} = \lim_{\Delta x \to 0}\frac{\Delta u + \Delta v}{\Delta x}= \lim_{\Delta x \to 0}\Big{[}\frac{\Delta u}{\Delta x}+\frac{\Delta v}{\Delta x}\Big{]}\\
                 &=\lim_{\Delta x \to 0}\frac{\Delta u}{\Delta x} + \lim{\Delta x \to 0}\frac{\Delta v}{\Delta x}= \frac{du}{dx}+\frac{dv}{dx}.
\end{align*}

\subsection{곱과 나눗셈 규칙}

덧셈과 상수 곱과 관련해 도함수를 구하는 규칙을 알아보았다. 이번에는 함수의 곱과 나눗셈에 대한 도함수를 구하는 규칙을 알아볼 것이다.\\
\[
    uv \text{ and } \frac{u}{v}, \text{ where u and v are differentiable function of x.}
\]
예로 $u=x^3, v=x^4$라 하자. 이 때, $uv$의 도함수는 각 함수의 도함수의 곱인 $\frac{d}{dx}(uv) = 3x^2 * 4x^3 = 12x^5$ 가 아닌, $\frac{d}{dx}(uv) = 7x^6$ 이다. 
다음은 두 함수의 곱으로 생성된 함수의 도함수 도출 증명이다. 
\begin{align*}
    &\textbf{Let } y=uv\\
    &y+\Delta y = (u+\Delta u)(v+\Delta v) = uv + u\Delta v + v\Delta u + \Delta u \Delta v\tag{1}\\
    &\Delta y = (y+\Delta y)- y = u\Delta u + v\Delta v + \Delta u\Delta v\\
    &\frac{\Delta y}{\Delta x} = u\frac{\Delta v}{\Delta x} + v\frac{\Delta u}{\Delta x} + \Delta u\frac{\Delta v}{\Delta x}\\
    &\text{Taking limits as } \Delta x \to 0 \text{ then, }\\
    &\frac{dy}{dx} = u\frac{dv}{dx} + v\frac{du}{dx} + 0\frac{dv}{dx}\tag{2}\\
    &\text{Thus, }\\
    &\frac{d}{dx}(uv) = u\frac{dv}{dx} + v\frac{du}{dx}.
\end{align*}
위 $(1)$은 $\Delta y = \Delta (uv)$이고, 각각의 변화량을 고려하기 때문에 $y+\Delta v = (u+\Delta u)(v+\Delta v)$이다. 그리고 $(2)$에서 $\Delta x \to 0$일 때, $\Delta u \to 0, \Delta v \to 0$이므로 나머지 항이 라이프니츠 표현식을 따라 $dy/dx$가 되며, 3항의 $\Delta u$가 $0$이 되는 것이다.

\end{document}