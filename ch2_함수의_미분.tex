## 함수의 미분 
# THE DERIVATIVE OF A FUNCTION 

## Contents
1. introduction  
2. 접선  
2.1 접선  
2.2 접선의 기울기를 구하기   
2.3 delta notation 1  
2.4 미분의 정의   
2.5 delta notation 2  
2.6 임의의 점에서 미분이 가능하지 않을 때  
2.7 속도와 변화율  
2.8Last 극한과 연속 함수  

## introduction  
`calculus`는 보통 크게 두 파트로 나뉜다. 첫 째는 `미분differential calculus`이고, 둘 째는 `적분integral calculus`이다.  

두 파트 모두 자신들만의 용어와 난해한 기호들 그리고 자신들만의 연산법이 있다. 그래서 이들을 배울 땐, 마치 각각의 새로운 언어를 배우는 것과 유사하다.  
하지만, `calculus`의 기초 영역과 응용 영역을 아우르는 핵심적인 질문은 다음 두가지 문제로 기술할 수 있다.

> **PROBLEM 1** 미분의 기저 문제는 *기울기의 문제* 이다. 그래프의 임의의 점 **P** 에 접하는 선의 기울기를 연산하는 것이다.  
> **PROBLEM 2** 적분의 기저 문제는 *넓이의 문제* 이다. 두 점 **x = a, x = b** 사이의 넓이를 연산하는 것이다.  

# 2. 도함수 
## 2.1 접선

원에 대한 접선은 비교적 이해하기 쉽다. 원에 존재하는 단 하나의 점과 교차하는 접선이다. 두 점에 교차하거나 아예 교차하지 않으면 접선이 아니다.  
단, 곡선에 대해서는 위 표현이 달갑지 않을 수 있다. 왜냐하면 임의의 곡선 위의 점 P에 대한 접선이 곡선의 다른 점과 만날 수도 있기 때문이다.  

따라서, 접선을 설명하기 위한 비교적 적절한 표현은 다음과 같다.  

> 곡선 $y = f(x)$ 가 있다고 하자. 곡선 위의 점 P에 대한 접선을 구한다고 하자. 곡선 위의 임의의 점 Q와 P를 가로지르는 선분을 긋자. 이 선분을 $\overline{PQ}$ 라 하자. 이때, Q를 곡선을 따라 P로 가까이 이동 시킬 때, 곡선 P에 접하는 접선을 구할 수 있다.  

## 2.2 접선의 기울기를 구하기  

곡선 위의 점 P가 다음을 만족한다고 하자. $P=(x_0,y_0)$    
그리고, 곡선 위의 다른점 Q가 다음을 만족한다고 하자. $Q=(x_1,y_1)$  

이 때, 선분 $\overline{PQ}$는 다음을 만족한다.  

$$m_{secant} = \text{slope of }\overline{PQ} = \frac{y_1-y_0}{x_1-x_0}\tag1$$

점 Q를 움직여, $x_1$을 $x_0$에 근접한다고 하자. 선분은 접선 $P$와 닮아 가는 것을 직관적으로 알 수 있다. 이러한 행위를 다음과 같이 표현할 수 있다.

$$m = \lim_{Q \to P} m_{sec} = \lim_{x_1 \to x_0} \frac{y_1 - y_0}{x_1 - x_0}\tag2$$  

$\lim_{x_1 \to x_0}$이란, $x_1$가 $x_0$에 점점 다가간다는 뜻이다. 이에 착안하여, 단순히 $x_1$ 에 $x_0$을 대입할 수는 없다. 왜냐하면 분모 ${x_1-x_0}$이 $0$이 되기 때문이다.

이로써 다음을 명심해야한다. $x_1$이 꾸준히 $x_0$에 가까워지지만, 여전히 $x_1$과 $x_0$ 사이에는 명확한 구별이 있어야한다는 점이다. 이를 꾸준히 연산하면 분모$y_1-y_1$와 분자$x_1-x_0$이 어떤 작은 수가 된다는 것은 분명하지만, 분수의 결과 값이 어떤 게 될지는 명확하지 않다.  

만약 점 $P$와 $Q$가 다음 함수 $y = x^2$ 위에 있다고 하자. 그러면 다음을 따른다.

$$m_{sec}=\frac{y_1-y_0}{x_1-x_0}=\frac{x_1^2-x_0^2}{x_1-x_0}=x_1+x_0\tag3$$

따라서, $y=x^2$ 함수에서 수식 $(2)$를 다음과 같이 표현할 수 있다.

$$m = \lim_{x_1 \to x_0}\frac{y_1 - y_0}{x_1 - x_0} = \lim_{x_1 \to x_0}\frac{x_1^2-x_0^2}{x_1-x_0} = \lim_{x_1 \to x_0}({x_1 + x_0})$$

이 때, $x_1$이 $x_0$에 근접하면 $x_1 + x_0$은 $x_0 + x_0= 2x_0$과 매우 매우 근접해 질것이다.

따라서, 점 $P$에서의 접선은 다음과 같다.  

$$m = 2x_0\tag4$$

## 2.3 delta notation 1 

`delta notation`은 $\Delta x$과 같이 표기한다. $\Delta$는 그리스 문자로 영어 알파벳 $d$를 뜻한다. 이 때 $d$는 `difference`를 뜻하며 두 변수간 값의 차이를 표현하기 위해 사용한다.

그러므로, 일반적으로 $\Delta x$는 다음을 뜻한다.  

$$\Delta x = x_1 - x_0 \tag1$$

다음 또한 만족한다.  

$$x_1 = x_0 + \Delta x \tag2$$

이를 이용해 $y=x^2$ 함수에서 임의의 정점 P에 대해 다시 한번 선분의 기울기를 표현해보자.

$$m_{sec}=\frac{x_1^2-x_0^2}{x_1-x_0}=\frac{(x_0+\Delta x)^2-x_0^2}{\Delta x} \tag3$$   

이 때, $(3)$의 식은 다음과 같이 풀어질 수 있다.  


$$
(x_0+\Delta x)^2 - x_0^2  
                = x_0^2+2x_0 \Delta x + (\Delta x)^2 - x_0^2   $$
$$\;\;\;\;\;\;\;\;\;= \Delta x(2x_0 + \Delta x)$$

따라서,  

$$ 
m_{sec} = 2x_0 + \Delta x 
$$ 

접선의 기울기를 표현할 때 사용한 극한 $\lim_{x_1 \to x_0}$는 $\lim_{\Delta x \to 0}$와 동치이다. 

마지막으로, 접선의 기울기는 다음과 같이 표현할 수 있다.

$$
m = \lim_{\Delta x \to 0}(2 x_0 + \Delta x) = 2x_0
$$


이제 까지 나온 표현을 모두 정리하면 다음과 같다. 

1. 선분의 기울기 


$$ m_{sec} = \frac{f(x_1) - f(x_0)}{x_1 - x_0 }$$
$$\;\;\;\;\;\;\;\;\;\;\;\;\;= \frac{f(x_0 + \Delta x) - x_0 }{\Delta x}$$
2. 접선의 기울기

$$ m = \lim_{x_1 \to x_0}\frac{f(x_1) - f(x_0)}{x_1 - x_0 } $$
$$ = \lim_{\Delta x \to 0} \frac{f(x_0 + \Delta x) - x_0 }{\Delta x}
$$

이 때, 미분에서는 $x_0$에서의 접선 기울기라고 좀 더 명확하게 표현하기 위해 다음과 같이 사용한다. 

$$
f'(x_0) = \lim_{\Delta x \to 0} \frac{f(x_0 + \Delta x) - f(x_0) }{\Delta x}
$$

## 2.4  도함수의 정의  

함수 $f(x)$가 주어졌을 때, 그것의 도함수 $f'(x)$가 정의역 $x$에서 갖는 값은 다음과 같다.    

$$
f'(x) = \lim_{\Delta x \to 0} \frac{f(x + \Delta x) - f(x) }{\Delta x}\tag1
$$

도함수를 설명할 때 곡선의 임의의 점에 대한 접선의 기울기의 집합이라고 표현할 수 있다. 그러나 반드시 곡선과 같은 그래프를 머리에 떠올려 접선을 그어보지 않아도 된다.  
여전히 기하학을 이용한 사고방식이 도함수의 이해에 큰 도움이 되지만, 다음 챕터에서는 또 다른 접근법을 알아볼 것이다.

도함수 $f'(x)$를 구하는 데엔 다음 세가지 방식을 따른다. 이 것을 `three-step rule`이라 하겠다. 앞서 다루었던 내용이므로 다시 한번 기억을 다듬어보는 것도 좋겠다.  

>STEP 1 변화율을 구한다. $f(x+\Delta x) - f(x)$   
>STEP 2 $\Delta x$로 나누어 평균 변화율을 구한다. $\frac{f(x+\Delta x)-f(x)}{\Delta x}$  
>STEP 3 $\lim_{\Delta x \to 0}$을 이용하여 평균 변화율의 극한값을 구한다.  

**Example 1** Find $f'(x)$ if $f(x) = x^3$  

**STEP 1**

$$
 f(x + \Delta x) - f(x) = (x+ \Delta x)^3 - x^3 
 $$
 $$
 = x^3 + 3x^2\Delta x + 3x(\Delta x)^2 + (\Delta x)^3 - x^3
 $$
 $$
 = 3x^2\Delta x + 3x(\Delta x)^2 + (\Delta x)^3
 $$
 $$
 = \Delta x(3x^2 + 3x\Delta x + (\Delta x)^2)
 $$

**STEP 2**

$$
\frac{f(x+\Delta x)- f(x)}{\Delta x} = 3x^2 + 3x\Delta x + (\Delta x)^2
$$

**STEP 3**

$$
f'(x) = \lim_{\Delta x \to 0}(3x^2 + 3x\Delta x + (\Delta x)^2) = 3x^2
$$

함수 $f(x) = \frac{1}{x}$ 와 $f(x) = \sqrt{x}$에 대해서 도함수도 구하는 절차는 독자에게 맡긴다.   
각각의 도함수는 $f'(x) = -\frac{1}{x^2}$과 $f'(x) = \frac{1}{2\sqrt{x}}$이다. 

결과 도함수의 이해를 돕기위해 간략히 쓴다. 도함수 $f'(x) = -\frac{1}{x^2}$는 $\forall x(x \ne 0)$에서 항상 음수이다. 그리고 $x$가 $0$에 근접할 수록 도함수 $f'(x)$의 매우 큰 수로 발산하는 것을 알 수 있다. 이는 $x$가 $0$에 근접할수록 접선이 매우 가팔라 진다는 것이다. $x$의 절대값이 커질수록 $f'(x)$는 $0$에 수렴하며, 접선이 수평선에 가까워진다는 것이다.   

## 2.5 delta notation 2  

수학에 있어 표기법은 굉장히 중요한 역할을 한다. 좋은 표기법은 좋은 방향성을 제시해주며, 나쁜 표기법은 쉬운 표현도 어렵게 만드는 법이다.  

우리는 함수 $f(x)$의 도함수를 $f'(x)$라고 표기하였다. 도함수는 함수 $f(x)$에서 도출되었지만 정의역 $x$에 대해선 새로운 함수임을 강조하기 위해 $f'(x)$라는 표기를 쓰므로, 그 의미를 표현하는데 적절하다고 할 수 있다. 종종, 함수의 치역 변수인 $y$를 써서 $f'(x)$를 $y'$로 쓰기도 한다.  

도함수를 표현하는데에 $f'(x)$의 사용은 단점도 있다. 도함수가 변화율을 기반으로 도출되었는지 명확하게 드러내주지 않는다. 

앞으로 소개할 수학자 `Leibniz`의 표기법은 도함수를 이해하는데 적절히 도움이 된다. 

도함수의 분자인 $y$의 변화율을 다음과 같이 표현하자.  

$$\Delta y = {f(x+\Delta x) - f(x)}$$

이 때, 도함수는 다음과 같이 표현한다.  

$$ 
\frac{dy}{dx} = lim_{\Delta x \to 0}\frac{\Delta y}{\Delta x}\tag2$$

$y = f(x)$이므로 다음과 같이 표현할 수 있다.  

$$
\frac{dy}{dx} =\frac{df(x)}{dx} = \frac{d}{dx}f(x) = f'(x)
$$

$\frac{dy}{dx}$는 $\lim$표현과 $\Delta$을 생략하는 대신 $d$로 대체한 것이다.   
읽는 사람으로 하여금 $\Delta y/\Delta x$ 변화율에서 $\Delta x\to 0$를 수행한다는 것을 빠르게 이해할 수 있어 이점이 있다.  

결론적으로, 정의역 위의 임의의 점 $x=a$에서의 도함수 값은 다음과 같이 표현할 수 있다.

$$
f'(a) = \Big( \frac{dy}{dx} \Big)_{x=a}  = \frac{dy}{dx} \Big|_{x=a}
$$

## 2.6 임의의 점에서 미분이 가능하지 않을 때

임의의 점에서 미분은 가능하지 않다. 그러면 그 점에서 접선은 존재하지 않는다는 것이다. 접선을 구하는 방식이 기억이 나는가?  

곡선 위의 점 $P$와 $Q$가 있을 때, $Q$를 점점 $P$로 가까이 곡선 위로 움직였다. `three-step rule`에서는 $3$번째 동작이다.

이 때, $Q$를 점점 $P$로 가까이 움직일 때 평균변화율의 좌극한과 우극한이 같지 않다. 아래 함수를 보자.

$$f(x)=|x-1|\tag1$$

위 절대값을 조건 방정식으로 표현하면 다음과 같다.  

$$|x-1| = \begin{cases}
   x-1 &\text{if } x>1 \\
   -x+1 &\text{if } x<1
\end{cases}$$

V 형태의 그래프가 된다. 아래는 $(1)$ 함수의 도함수이다. 

$$f'(1)=\lim_{\Delta x \to 0}\frac{f(1 + \Delta x)-f(1)}{\Delta x}=\lim_{\Delta x \to 0}\frac{|\Delta x|}{\Delta x}$$

아래는 평균변화율의 우극한이다.  

$$\lim_{\Delta x \to 0+}\frac{|\Delta x|}{x} = \lim_{\Delta x \to 0+}\frac{\Delta x}{\Delta x} = 1$$

아래는 좌극한이다. $\Delta x$는 음수의 값에서 부터 $0$으로 다가오니, 절대값 기호를 벗음과 동시에 음수 기호를 붙여줘야한다.

$$\lim_{\Delta x \to 0-}\frac{|\Delta x|}{x} = \lim_{\Delta x \to 0-}\frac{-\Delta x}{\Delta x} = -1$$

이로써 $x=1$에서의 접선 기울기가 왼쪽에서 좁혀질 때와 오른쪽에서 좁혀질 때가 다르므로 $1$에서의 미분은 가능하지 않다.  

함수 하나를 더보자.  

$$f(x) = |x| + x$$

이때, $x = 0$에서 미분 가능하지 않음을 보일 수 있겠는가?  

$$
f'(0) = \lim_{\Delta x \to 0}\frac{f( 0 + \Delta x) - f(0)}{\Delta x} = \lim_{\Delta x \to 0}\frac{|\Delta x| + \Delta x}{\Delta x}
$$

나머지 증명은 독자들에게 맡긴다.   

## 2.7 속도와 변화율  

실생활에서 쓰일 수 있는 속도 문제에 대해서 이야기해보자.  
일차원 직선위에 어느 물체가 움직이고 있다고하자. 매 초마다 그 물체가 어디에 있는지 정확히 알고 있다고 하자. 그러면 시간에 대한 위치 함수는 다음과 같다.  

$$
position = f(time)
$$

편의를 위해 $position$의 $s$를, $time$의 $t$를 쓰자.

$$
s= f(t)
$$

낙하 운동하는 물체의 시간에 대한 위치 함수가 $s=16t^2$이라 하자. 그렇다면 $t=5$일때, $s=400$이다. 만약 단위가 각각 $s(meter), t(seconds)$라 할때, $t=5sec$일때, $s=400m$이다. 만약 $400m$높이에서 떨어뜨렸다면, $5sec$에 지면에 도달한다.  

이 예제에서, $t=1sec \sim 3sec$ 동안 이동 구간은 $s=16m \sim 144m$이다. 그럼 $1sec \sim 3sec$ 구간에서 평균 속도는 ${\Delta s}/{\Delta t} = 64m/s$ 이다. ${\Delta t}$를 점점 줄여나갈 수록 $t = a$에서의 속도를 구할 수 있을 것이다.

$$
v=\frac{ds}{dt} = \lim_{\Delta t \to 0}\frac{\Delta s}{\Delta t}
$$

$$
=\lim_{\Delta t \to 0}\frac{16(t+\Delta t)^2 - 16t^2}{\Delta t}
$$

$$
=\lim_{\Delta t \to 0}(32t + 16\Delta t) = 32t
$$

따라서, 매 초 속도는 32의 배수로 증가한다. 물체가 5초일 때 바닥에 닿는데 그 때의 속도는 $165m/s$ 이다.  
이 $v = 32t$를 표현할 때는 다음과 같이 표기한다.

$$
v = m/s = 32s
$$

$$
m/s^2 = 32 \text{ or  } a = 32m/s^2 
$$

이 때 $a$는 가속도$acceleration$의 $a$이다. 

이번에는 다른 유형의 변화율에 대해서 이야기 해보고싶다.  

반지름이 $r$인 원을 생각해보자. 이 때, 반지름의 변화에 따른 원 넓이의 변화율은 무엇일까?

$$
Area = \pi r^2 
$$

$$
\frac{d}{dr}Area = \lim_{\Delta r \to 0}\frac{f(r + \Delta r) - f(r)}{\Delta r} = \lim_{\Delta r \to 0}\frac{2\pi r\Delta r + (\Delta r)^2}{\Delta r}
$$

따라서, 반지름 변화에 따른 원 넓이의 변화율은 다음과 같다.  

$$
\frac{d}{dr}Area = \lim_{\Delta r \to 0}(2\pi r\Delta r+\Delta r) = 2\pi r
$$

반지름이 커지면 커질수록, 넓이의 변화도 점점 커진다.  
변화율과 관련한 몇 몇 재밌는 문제들을 `ch2 problem section`에 남겨두었다.  


## 2.8Last 극한과 연속 함수

아래 표현식을 보자. 

$$
\lim_{x \to a}f(x) = L\tag1
$$

$x \to a$로 갈 때, 함수 $f(x)$의 값은 $L$이라는 것이다. 이는 $f(a)=L$을 뜻하는 것은 아니다.  
처음 접선의 기울기 값을 구할 때 평균 변화율 $\Delta y / \Delta x$에서 $\Delta x \to 0$을 수행할 때에도 $\Delta x$와 $0$ 간에는 여전히 구별할만한 어떤 것이 있다고 하였다.  
또, 다음 절대값 함수 $y=|x|$의 $x=0$에서의 평균 변화율 $|\Delta x|/\Delta x$에서 $\Delta x \to 0$가 하나의 값으로 결정할 수 없는 것과도 상관이 있다. $|\Delta x|$가 음수 또는 양수 방향에서 구별할만한 어떤 것을 가지는가에 따라 절대값 기호가 달라지기 때문이다.   

더 적극적인 설명을 돕기 위해 다음 함수를 제시해 본다.  

$$
y=\frac{2x^2+x}{x}\tag2 
$$

위 식은 다음과 같이 쓸수 있다.

$$
y=2x+1, \text {where } x\ne0\tag3 
$$

$(2)$에서 분모가 0이 될 수없으므로, $(3)$은 적절한 변환이다. $x \to 0$일 때, $y$의 값은 어떻게 변하겠는가?  

$$
y = \frac{2}{1000}+1, \text{ when } x = \frac{1}{1000} 
$$

$$
y = \frac{2}{10000}+1, \text{ when } x = \frac{1}{10000} 
$$

$$
y = \frac{2}{100000}+1, \text{ when } x = \frac{1}{100000}
$$

$$
\text {.......}
$$

반복하다보면 $y \to 1$임을 알 수 있다. 위의 수식을 좀 더 일반적으로 표현하기 위해 다음의 표현을 쓰자.  
$\epsilon$이 $0$이 아닌 어떤 양의 값이라고 해보자. 그리고 다음을 만족하는 $\delta$가 있다고 하자.   

$$
\delta = (1/2)*\epsilon
$$

$x$와 $0$의 거리가 $\delta$보다 작다면, $f(x)$와 $1$의 거리가 $\epsilon$보다 작다.    

$$
\text{ if }\;  |x| \lt ( \delta = \frac{1}{2} * \epsilon), \;\text{ then } ( | f(x) - 1| = |2x|) < \epsilon 
$$

위 표현식은 $x \to 0$행위에 대해 좀 더 엄밀하게 표현해준다. $\delta \ne 0, x \ne 0$을 만족하면서 $x$와 $0$의 거리가 굉장히 작아진다.   
이 때, 실수 체계에서 굉장히 작은 $\delta$를 꾸준히 제시할 수 있으며, $| f(x) -1 |, \epsilon$ 값도 작아지는 것을 알 수 있다. 

식 $(1)$의 $\lim_{x \to a}f(x) = L$에 대해 `epsilon-delta`표현을 사용하면 아래와 같다.  

$$
\text {if } 0 < |x-a| <\delta
$$


$$
\text {then }
|f(x) - L| < \epsilon, x \in \R
$$

이로써 극한값이란, 함수의 임의의 점에서 반드시 함수값을 가지지 않고 있어도 되며 $x$가 임의의 점으로 수렴할수록 어떠한 값에 꾸준히 수렴해나가는 형태를 띄고 있으면 된다.  `(epsilon-delta 부분은 추후에 수정 부탁)`  

다음은 극한 값을 구하는 규칙을 짤막히 정리한 것이다.  

$$
\lim_{x \to a}x = a
$$

$$
\lim_{x \to a}c = c, \text { where  } c \text { is constant.  }
$$

그리고 $\lim_{ x \to a } f(x) =L, \lim_{x \to a}g(x) = M$이라면 다음을 만족한다.

$$
\lim_{x \to a}[f(x) + g(x)] = L + M
$$

$$
\lim_{x \to a}[f(x) - g(x)] = L - M
$$

$$
\lim_{x \to a}[f(x) * g(x)] = L * M
$$

$$
\lim_{x \to a}\frac{f(x)}{g(x)} = \frac{L}{M}, \text {where }  M \ne 0
$$

연속이란 다음을 만족한다.  

$$
\lim_{x \to a}f(x) = f(a), x \in \R
$$

연속 함수란 다음을 만족한다.  

$$
\lim_{x \to a}f(x) = f(a), \forall x
$$

